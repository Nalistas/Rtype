\chapter{Préanbule des contes d\textquotesingle{}Ul}
\hypertarget{md__2home_2solype_2delivery_2current__days_2Mannheim_2back_2server_2public_2lore_2preambule}{}\label{md__2home_2solype_2delivery_2current__days_2Mannheim_2back_2server_2public_2lore_2preambule}\index{Préanbule des contes d\textquotesingle{}Ul@{Préanbule des contes d\textquotesingle{}Ul}}
\label{md__2home_2solype_2delivery_2current__days_2Mannheim_2back_2server_2public_2lore_2preambule_autotoc_md0}%
\Hypertarget{md__2home_2solype_2delivery_2current__days_2Mannheim_2back_2server_2public_2lore_2preambule_autotoc_md0}%
 Les textes anciens sont assez explicites quant à la distinction essentielle entre chouette et hibou. « Le hibou possède des aigrettes, la chouette pas. » peut-\/on ainsi lire dès le premier chapitre du {\itshape Livre d’\+Ul}. Plus loin, le texte antique précise \+: « C’est comme des sourcilles qui forment des cornes, et la chouette, elle en a pas. ». Assez descriptif, ce n’est pas dans le livre d’\+Ul que l’on trouve les avertissements les plus explicites à propos de toute confusion. Dans sa brillante Lamentation à la Lune, traduite du revlij par sa disciple, Lulu la Sixième Murmurante de Nuit écrivait \+: « Celui qui confond chouette et hibou Finira suffoqué dans la boue, Celle qui confond hibou et chouette Sera par le fer réduite en miettes. ». Le sinistrement célèbre {\itshape Code de conduite général}, dont l’application tyrannique coûta la vie à bien des innocents, a même été jusqu’à punir sévèrement cette erreur \+: « La mort par lapidation et chatouilles de la plante des pieds attend tout individu, qui, parlant d’une chouette, se trouvait en vérité à parler d’un hibou, et réciproquement. Al.\+II, art.\+VIII».

Extrait de {\bfseries{La Chouette et le Hibou}}, Histoire d’une distinction fondamentale de Tuoteche Lwo, première Année du Rossignol après la Petite Chute. 
\chapter{tailwindcss/nesting}
\hypertarget{md__2home_2solype_2delivery_2current__days_2trello_2front_2node__modules_2tailwindcss_2src_2postcss-plugins_2nesting_2README}{}\label{md__2home_2solype_2delivery_2current__days_2trello_2front_2node__modules_2tailwindcss_2src_2postcss-plugins_2nesting_2README}\index{tailwindcss/nesting@{tailwindcss/nesting}}
\label{md__2home_2solype_2delivery_2current__days_2trello_2front_2node__modules_2tailwindcss_2src_2postcss-plugins_2nesting_2README_autotoc_md13051}%
\Hypertarget{md__2home_2solype_2delivery_2current__days_2trello_2front_2node__modules_2tailwindcss_2src_2postcss-plugins_2nesting_2README_autotoc_md13051}%
 This is a Post\+CSS plugin that wraps \href{https://github.com/postcss/postcss-nested}{\texttt{ postcss-\/nested}} or \href{https://github.com/csstools/postcss-plugins/tree/main/plugins/postcss-nesting}{\texttt{ postcss-\/nesting}} and acts as a compatibility layer to make sure your nesting plugin of choice properly understands Tailwind\textquotesingle{}s custom syntax like {\ttfamily @apply} and {\ttfamily @screen}.

Add it to your Post\+CSS configuration, somewhere before Tailwind itself\+:


\begin{DoxyCode}{0}
\DoxyCodeLine{//\ postcss.config.js}
\DoxyCodeLine{module.exports\ =\ \{}
\DoxyCodeLine{\ \ plugins:\ [}
\DoxyCodeLine{\ \ \ \ require('postcss-\/import'),}
\DoxyCodeLine{\ \ \ \ require('tailwindcss/nesting'),}
\DoxyCodeLine{\ \ \ \ require('tailwindcss'),}
\DoxyCodeLine{\ \ \ \ require('autoprefixer'),}
\DoxyCodeLine{\ \ ]}
\DoxyCodeLine{\}}

\end{DoxyCode}


By default, it uses the \href{https://github.com/postcss/postcss-nested}{\texttt{ postcss-\/nested}} plugin under the hood, which uses a Sass-\/like syntax and is the plugin that powers nesting support in the \href{https://tailwindcss.com/docs/plugins\#css-in-js-syntax}{\texttt{ Tailwind CSS plugin API}}.

If you\textquotesingle{}d rather use \href{https://github.com/csstools/postcss-plugins/tree/main/plugins/postcss-nesting}{\texttt{ postcss-\/nesting}} (which is based on the work-\/in-\/progress \href{https://drafts.csswg.org/css-nesting-1/}{\texttt{ CSS Nesting}} specification), first install the plugin alongside\+:


\begin{DoxyCode}{0}
\DoxyCodeLine{npm\ install\ postcss-\/nesting}

\end{DoxyCode}


Then pass the plugin itself as an argument to {\ttfamily tailwindcss/nesting} in your Post\+CSS configuration\+:


\begin{DoxyCode}{0}
\DoxyCodeLine{//\ postcss.config.js}
\DoxyCodeLine{module.exports\ =\ \{}
\DoxyCodeLine{\ \ plugins:\ [}
\DoxyCodeLine{\ \ \ \ require('postcss-\/import'),}
\DoxyCodeLine{\ \ \ \ require('tailwindcss/nesting')(require('postcss-\/nesting')),}
\DoxyCodeLine{\ \ \ \ require('tailwindcss'),}
\DoxyCodeLine{\ \ \ \ require('autoprefixer'),}
\DoxyCodeLine{\ \ ]}
\DoxyCodeLine{\}}

\end{DoxyCode}


This can also be helpful if for whatever reason you need to use a very specific version of {\ttfamily postcss-\/nested} and want to override the version we bundle with {\ttfamily tailwindcss/nesting} itself. 